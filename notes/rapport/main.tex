\documentclass[12pt, a4paper]{memoir} % for a short document
\usepackage[french,english]{babel}

\usepackage [vscale=0.76,includehead]{geometry}                % See geometry.pdf to learn the layout options. There are lots.
%\geometry{a4paper}                   % ... or a4paper or a5paper or ...
%\geometry{landscape}                % Activate for for rotated page geometry
%\OnehalfSpacing
% \setSingleSpace{1.05}
%\usepackage[parfill]{parskip}    % Activate to begin paragraphs with an empty line rather than an indent


%===================================== packages
\usepackage{lipsum}
\usepackage{graphicx}
\usepackage{amsmath}
\usepackage{fullpage}
\usepackage{mathptmx} % font = times
\usepackage{helvet} % font sf = helvetica
\usepackage[utf8]{inputenc}
\usepackage{relsize}
\usepackage[T1]{fontenc}
\usepackage{tikz}
\usepackage{booktabs}
\usepackage{textcomp}%textquotesingle
\usepackage{multirow}
\usepackage{pgfplots}
\usepackage{url}
\usepackage{footnote}
\usepackage{hyperref}
% MINE

\usepackage{listings}
\usepackage{color}
\usepackage{float}


%============================================
\usetikzlibrary{arrows,shapes,positioning,shadows,trees}
\makesavenoteenv{tabular}
\makesavenoteenv{table}
%==============================================
\def\checkmark{\tikz\fill[scale=0.4](0,.35) -- (.25,0) -- (1,.7) -- (.25,.15) -- cycle;}
%Style des têtes de section, headings, chapitre
\headstyles{komalike}
\nouppercaseheads
\chapterstyle{dash}
\makeevenhead{headings}{\sffamily\thepage}{}{\sffamily\leftmark}
\makeoddhead{headings}{\sffamily\rightmark}{}{\sffamily\thepage}
\makeoddfoot{plain}{}{}{} % Pages chapitre.
\makeheadrule{headings}{\textwidth}{\normalrulethickness}
%\renewcommand{\leftmark}{\thechapter ---}
\renewcommand{\chaptername}{\relax}
\renewcommand{\chaptitlefont}{ \sffamily\bfseries \LARGE}
\renewcommand{\chapnumfont}{ \sffamily\bfseries \LARGE}
\setsecnumdepth{subsection}


% Title page formatting -- do not change!
\pretitle{\HUGE\sffamily \bfseries\begin{center}}
\posttitle{\end{center}}
\preauthor{\LARGE  \sffamily \bfseries\begin{center}}
\postauthor{\par\end{center}}
\newcommand{\jury}[1]{%
\gdef\juryB{#1}}
\newcommand{\juryB}{}
\newcommand{\session}[1]{%
\gdef\sessionB{#1}}
\newcommand{\sessionB}{}
\newcommand{\option}[1]{%
\gdef\optionB{#1}}
\newcommand{\optionB} {}

\renewcommand{\maketitlehookd}{%
\vfill{}  \large\par\noindent
\begin{center}\juryB \bigskip\sessionB\end{center}
\vspace{-1.5cm}}
\renewcommand{\maketitlehooka}{%
\vspace{-1.5cm}\noindent\includegraphics[height=12ex]{pics/logo-uga.png}\hfill\raisebox{2ex}{\includegraphics[height=14ex]{pics/logoINP.png}}\\
\bigskip
\begin{center} \large
Master of Science in Informatics at Grenoble \\
Master Informatique \\
Specialization \optionB  \end{center}\vfill}
% =======================End of title page formatting

\option{Graphics, Vision and Robotics}
\title{Procedural Stylization} %\\\vspace{-1ex}\rule{10ex}{0.5pt} \\sub-title}
\author{Isnel Maxime}
\date{June 2019} % Delete this line to display the current date
\jury{
Research project performed at MAVERICK \\\medskip
Under the supervision of:\\
Romain Vergne\\
Joëlle Thollot\\\medskip
Defended before a jury composed of:\\
James Crowley\\
Dominique Vaufreydaz\\
}
\session{June \hfill 2019}
\setcounter{tocdepth}{4}
\setcounter{secnumdepth}{4}

%%% BEGIN DOCUMENT
\begin{document}

\selectlanguage{English} % french si rapport en français
\frontmatter
\begin{titlingpage}
\maketitle
\end{titlingpage}

%\small
\setlength{\parskip}{-1pt plus 1pt}

\renewcommand{\abstracttextfont}{\normalfont}
\abstractintoc
\begin{abstract}


Stylization is the action of representing an object in 3 dimensions in a simple and decorative way. This 3D scene stylization uses images or filtering. It is often subject to temporal coherence problems. These problems are the subject of this report.

Time coherence problems are present in animated scenes. For example, elements of the image appear and disappear in the next image, brushstrokes do not follow the movement of the object, pencil strokes become larger as you approach the object, etc.

In this report, we describe a method that aims to minimize these temporal consistency problems while maintaining control over the stylization of the object. To do this, our method uses the principles of stylization in an image based on the collage of images containing brushstrokes, hair, dots, leaves, etc. We also use 3D textures to control the movement of these images on the screen.

\end{abstract}
\abstractintoc



\renewcommand\abstractname{R\'esum\'e}
\begin{abstract} \selectlanguage{French}

Styliser est l’action de représenter un objet en 3 dimensions de manière simple et décorative. Cette stylisation de scène 3D utilise des images ou du filtrage. Elle est souvent sujet à des problèmes de cohérence temporelle. Ces problèmes font l’objet de ce rapport. \newline

Les problèmes de cohérence temporelle sont présents dans les scènes animées. Par exemple, des éléments de l’image apparaissent et disparaissent à l’image suivante, les coups de pinceau ne suivent pas le mouvement de l’objet, les coups de crayon deviennent plus gros quand on se rapproche de l’objet, etc. \newline

Dans ce rapport, nous décrivons une méthode qui vise à minimiser ces problèmes de cohérence temporelle tout en gardant le contrôle sur la stylisation de l'objet. Pour cela, notre méthode utilise les principes de stylisation dans une image à base de collage d'images contenant des coups de pinceau, des cheveux, des points, des feuilles, etc. Nous utilisons également des textures 3D afin de contrôler le mouvement de ces images sur l'écran.

\end{abstract}

\renewcommand\abstractname{Acknowledgement}
\begin{abstract}
    \selectlanguage{French}

    Avant tout développement, il me semble important de remercier toutes les personnes ayant contribué au bon déroulement de mon stage et qui ont rendu cette expérience riche, variée et unique. \newline

     Je voudrais tout d'abord remercier mes tuteurs de stages, Joëlle Thollot et Romain Vergne, pour leur patience, leurs explications et leur disponibilité qu’ils m’ont accordé. Ils m'ont partagé leurs savoirs sur le rendu non-photoréaliste et rendu graphique, leurs compétences de gestion d'un projet de recherche et leurs connaissances techniques. \newline

     Je voudrais aussi remercier toute l'équipe MAVERICK pour leur chaleureux accueil. Il fut un réel plaisir de travailler avec toutes ces personnes attentionnées et il fut plaisant et flatteur d’être considérée comme un membre de cette équipe. Pour finir, je voudrais remercier l'INRIA pour m'avoir procuré les ressources nécessaires pour mener à bien ce stage et cette recherche.


\end{abstract}

\selectlanguage{English}

\cleardoublepage

\tableofcontents* % the asterisk means that the table of contents itself isn't put into the ToC
\normalsize

\mainmatter
\SingleSpace
%==============================CHAPTERS==================
\chapter{Introduction}


\section{Background}

\section{Problem Statement}

The main problem of stylizing a 3D object in an animation is the \textit{temporal coherence}. The effect given by the stylization has to be kept if the object is moving, rotating and scaling. Many research have been done to solve this problem of \textit{temporal coherence} \cite{vergne_implicit_2011, benard_dynamic_2009, bleron_motion-coherent_2018}. This problem is three sections:

\subsection{Flatness}

The impression of drawing on a flat surface gives the \textit{flatness}. The stylization has a good \textit{flatness} is the image rendered has a good 2D appearence. In order to keep this effect the size and the distribution of the marks of your stylization has to be independant to the distance between the stylized object and the camera.

\subsection{Motion Coherence}

\textit{Motion coherence} is a correlation between the motion of marks and the motion of the 3D object. Bad \textit{Motion coherence} will give the impression to see the scene through semi-transparent layer of marks, this is called \textit{shower door} effect \cite{meier_painterly_1996}.

\subsection{Temporal continuity}


example \textit{Loving Vincent}

\chapter{Previous Work}

Image stylization has been around for years. Algorithms were created to automatise this desire to stylize. Some techniques use line extraction algorithm to then use convolution of points to make hand drawing styles. Hertzmann with his \textit{curve stroke} algorithm \cite{rosin_stroke_2013}  succeed to create images that look like a traditional painting with paintbrushes. To do so he computes many control pint on the original image to further place strokes. But these create a problem when we wanted to stylize videos because it treats frames independantly and so it creates bad \textit{motion continuity}. The movie \textit{Loving Vincent}\cite{LovingVincent} can illustrate what can happen in this case of bad \textit{motion continuity}.

Then some researches have be to propose a solution to this issue\cite{litwinowicz_processing_1997, hays_image_2004, bousseau_video_2007, lin_video_nodate}. The solution of Lin et \textit{al.} \cite{lin_video_nodate} is to create a segmentation manually of each key frame and then for each part of this segmentation they compute the motion. With this motion they adapt the stroke based rendering of the next frames. To have a watercolor stylization on a video Bousseau et \textit{al.} compute a texture advection to apply to the final image the wanted effect.

In our approach, the goal is to make stylized rendering of 3D objects. There are two moments in a pipeline rendering when we can stylize an object, the first is when we manipulate the vertices and the color of each triangle it is the \textit{object space}. The second is when we do the compositing with the textures that we have like shadow map, image filter, ... (manipulation of pixels of the screen) it is the \textit{image space} and also called \textit{screen space}.

\chapter{Realisation}


\section{Overview}

As explains above in the state of the art and in the figure\ref{tableau_comparatif} each approach has its advantages and its disadvantages. That is why in our solution we tried to take the better of the two worlds. We stylize the 3D scene in image space (screen space) but with all the information about the 3D object and the camera (camera matrices, position, normals, tangents, UV coordinates, distance from the camera). This solution permits to apply something like 2D images on the screen so have a good \textit{flatness} while keeping the information on the silhouettes, the orientation, the depth, etc. This solution permits also to easily integrate the stylizing of a scene in a pipeline rendering because it can be done at the end during the post-processing rendering pass. \newline

We chose to use mark based methods to stylize our scene because texture based methods in image space give a poor variety of styles as said in the work of Bénard et al.\cite{benard_dynamic_2009}. This mark based method implies to decide where in the image the splats will be drawn. In our problem, the goal is to anchor these splats with the objects in order to have the same motion for the splats and the object. This avoid the problem of \textit{shower door effect} and ensure the good \textit{motion coherence}. So we needed anchor points depending on the position of our object. Therefore in our approach, we used procedural noise\cite{perlin_improving_2002} as a texture of our 3D object. The procedural noises are easy to implement, fast to compute and easy to manipulate. Like every texture computed in object space, it has a good motion coherence. Each value different of zero of this texture represents an anchor point for a splat.


\section{Procedural noise and fractalization}

\begin{figure}
    \begin{center}
    \includegraphics[width=40mm, height=40mm]{images/PerlinNoise2d.png}
    \includegraphics[width=40mm, height=40mm]{images/GaborNoise2d.png}
    \includegraphics[width=40mm, height=40mm]{images/WorleyNoise2d.jpg}
    \end{center}
    \caption{Examples of procedural texture: \textit{left using perlin noise, middle using Gabor noise,right using worley noise}.}
    \label{procedural_texture}
\end{figure}

% Description of procedural noise

We compute procedural texture in order to create anchor points. Procedural noises are \textit{pseudorandom} gradient of grid point. In computer grapchis, they are usually used in 2d \ref{procedural_texture} as an image or in 3d for texturing a 3d object, in our case we use it in 3d. This texture are computed from procedural noise with a mathematical process. There exist many procedural noise such as Perlin noise, Worley noise, Gabor noise, Value noise, Gradient noise, etc. In order to map the procedural texture to the 3d object, we compute it with the vector position of each vertex of the object. The frequency of a noise control how many details there is in the texture. With a Worley noise, increasing the frequency will increase the number of black area in the texture.

% how we use it

\begin{figure}
    \begin{center}
    \includegraphics[scale=0.6]{images/noise/addition.png}
    \end{center}
    \caption{Usage of procedural texture to anchor splats: (left: splat image, middle: procedural teture, right: rendered image, bottom: color).}
    \label{procedural_noise_anchor}
\end{figure}

In our case we used the procedural texture to anchor the splat. For each pixel of the final image we "paste" a splat if the current pixel correspond to black pixel in the procedural texture the splat is not displayed (see Figure \ref{procedural_noise_anchor}). Thanks to this mechanism we can control the density of splat in the image. The value of our noise in the texture is between 0 and 1. The opacity vary according to this value of the procedural noise. One problem of working in image space is the aliasing, it creates some problem with \textit{temporal continuity} and \textit{motion coherence}. To reduce the aliasing in the procedural texture we make multiple samples with very small variation and we do an average to have the final value.

We add a threshold parameter to compute the noise in order to reduce the number of splat to display but especially to have small points in the procedural texture if not the splats convolve as you can see in the rendered image in the figure \ref{procedural_noise_anchor}.

% Worley

\begin{figure}
    \begin{center}
    \includegraphics[scale=0.2]{images/noise/worley_explain.png}
    \end{center}
    \caption{Worley noise in 2d.}
    \label{worley_explain}
\end{figure}

We mainly work with the Worley noise which is a cellular noise as you can see at the right in the figure \ref{procedural_texture}. To create a procedural texture from a Worley noise, we divide the image in cells (squares with the same size in 2d and cubes with the same size in 3d, red lines in the figure \ref{worley_explain}), then in each square (cube in 3d) we compute a point with a seeded random (white dot in the figure \ref{worley_explain}) and finally at each pixel of the texture we put as value the distance with the closest point computed previously with the random seed. Increase the frequency of this noise correspond to divide the image with smaller cells and so have more cells in the image. In our pipeline we use the inverse of this noise (1-worleynoise) in order to have the white pixels of the procedural texture as anchor points.

\begin{figure}
    \begin{center}
    \includegraphics[scale=0.3]{images/noise/worley3d_thresh_1.png}
    \includegraphics[scale=0.3]{images/noise/worley3d_thresh_2.png}
    \includegraphics[scale=0.3]{images/noise/worley3d_thresh_3.png}
    \end{center}
    \caption{Use of threshold in the case of Worley noise with the same frequency.}
    \label{worley_threshold}
\end{figure}

As we said before we use a threshold for the purpose of having an another control on the noise. With the Worley noise this threshold correspond to the "size" of the points. If we increase the threshold the points become smaller as you can see in the figure \ref{worley_threshold}. In order to facilate the control of this noise we adapt this threshold according to frequency and the distance from the camera. That means that the size of each point is constant if the frequency or/and the distance from the camera is changed. \newline

% Add fractalization


\textbf{Fractalization}

\begin{figure}
    \begin{center}
    \includegraphics[scale=0.3]{images/fractalization_principle.png}
    \end{center}
    \caption{Principle of fractalization \cite{benard_dynamic_2010}.}
    \label{fractalization_principle}
\end{figure}

\begin{figure}
    \begin{center}
    \includegraphics[scale=0.6]{images/fractal_explained.png}
    \end{center}
    \caption{Combination of the noises with the weight value in this case with 2 noises.}
    \label{fractalization_practical}
\end{figure}


Bénard et \textit{al.}\cite{benard_dynamic_2010} introduce the mechanism of fractalization on textures. This technique creates an impression of infinite zoom effect (like in this example: \href{https://www.shadertoy.com/view/XlBXWw?fbclid=IwAR1fU2JxQzXtks1ZcmVmzrHiv646G8w2gWceeiV-UToeFkAFMQ2NecbsGGs}{ShaderToy of Neyret Fabrice}). In our method of stylizing it improves the \textit{temporal continuity} because there is always display even if you get very close to the object and it also improves the \textit{flatness} impression because there is almost the same number of anchor points in the screen and their size is quasi-constant. This method alters the original pattern of the texture as you can see in the figure \ref{fractalization_principle} it can be an issue because some pattern cannot be fractalized (like the checkboard pattern) but it works well with stochastic textures. To creates this fractalization, they use textures at multiple frequencies (not necessary procedural texture) (see figure \ref{fractalization_principle}) and they combine them with the transparency and they overlap them according to the distance from the camera as you can see in the results of the figure \ref{fractalization_principle}. \newline


The figure \ref{fractalization_principle} show 4 noises (octaves) at different frequency it is a zooming cycle. As you can see, if the distance from the camera is close to 0 the frequency of the noises is high and if we move away from the object the frequency decrease. In the figure, we have: \texttt{2*frequency(octave 1) = frequency(octave 2)} and \texttt{2*frequency(octave 2) = frequency(octave 3)} and etc.  because in this case they divide the distance from the camera with a NON ! \textit{log2} so when the distance from camera is twice as large the frequency of the noise displayed is multiplied by 2. During the zoom, we modulate each octave with a weight. To compute the weight of each noise we use Gaussian interpolation centered on our current distance from the camera (see figure \ref{fractalization_practical}) with a sigma depending on the number of noises that we want to combine.







 % Bénard et \textit{al.}\cite{benard_dynamic_2010} use the same principle but with procedural textures. They create multiple noises with different frequency and combine them playing with transparency. Moreover, they overlap the noise to make an impression of infinite zoom effect (like in this example: \href{https://www.shadertoy.com/view/XlBXWw?fbclid=IwAR1fU2JxQzXtks1ZcmVmzrHiv646G8w2gWceeiV-UToeFkAFMQ2NecbsGGs}{ShaderToy}). With this method patterns of the texture have an almost constant size regardless of the size of the object but it can create small problems of \textit{temporal continuity}. In our method, we will use this technique of fractalization of a procedural noise. \newline

\section{Splatting}

\begin{figure}
    \begin{center}
    \fbox{\includegraphics[width=30mm, height=30mm]{images/splats/dot_splat.png}}
    \fbox{\includegraphics[width=30mm, height=30mm]{images/splats/hair_splat.png}}
    \fbox{\includegraphics[width=30mm, height=30mm]{images/splats/line_splat.png}}
    \fbox{\includegraphics[width=30mm, height=30mm]{images/splats/paint_splat.png}}
    \end{center}
    \caption{Example of what the splats can be.}
    \label{splat_examples}
\end{figure}

Artists draw on a flat surface that gives a good impression of \textit{flatness}. We use the same principle to stylize our scene. We put splats/marks directly on the image like an artist will do (illustrated in the figure \ref{procedural_noise_anchor}). In our solution, the user controls the splat image he can use anything. These marks can be leaves, hairs, dots, feathers, lines, paint brushes, etc. They can also be created from a procedural noise. The user has also the control on the size of these splats, he has a global control of the size and control on a specific part of the object through a texture. The rotation of the splats is also a parameter that the user can control. He can choose to rotate all the splat, for example, depending on the normals or depending on the tangents. \newline

Artists control how their marks are combined for example if he is doing painting he could want to have non-transparent marks in order to cover the marks behind the new one. He could want more transparency if for example he wants to do watercolorization. In our method, we take this into account during the blending of all marks. The user can control how many marks the stylization will mix and control the transparency (alpha blending). For example, the user can choose to only show the marks at the top of the flat surface (the closest to the camera) or he can choose to mix some marks in order to account some marks covered.


\section{Stylization}

\chapter{Practical implementation}




During our research on stylization, we used Gratin which a pipeline rendering software using \textit{OpenGL}. It permits to easily creates rendered images with \textit{GLSL} combining previous images as textures <insert screenshot of gratin>. As we said before every texture used from our contribution can be computed in parallel on GPU including the noises, the fractalization of the noises, the alpha blending and the splatting. Our method use only images as input so it can easily be integrated into every pipeline rendering. For example, if you have a pipeline rendering with \textit{path tracing} to compute the color of your scene you can use this rendered image as color input of our method without re-compute it.\newline

\begin{figure}
    \begin{center}
    \fbox{\includegraphics[width=60mm, height=60mm]{images/splatting/splatting4.png}}
    \end{center}
    \caption{Splatting principle: red dots are the anchor points.}
    \label{splatting_principle}
\end{figure}

For the splatting step, we draw on each pixel of the screen a square centered on the current pixel (see figure \ref{splatting_principle}). Every square is treated independently on the GPU before in the \textit{vertex shader} which manipulate the 4 vertices of the square which permits the resizing of the splat and permits the rotation of the splat. After this step of resizing and rotation, we pass in the \textit{fragment shader} which manipulate all pixels of splat. It is in this step that we will decide if we display the splat or not thanks to the procedural texture previously computed. \newline

\begin{figure}
    \begin{center}
    \fbox{\includegraphics[scale=0.8]{images/splatting/order_independant_transparency.png}}
    \end{center}
    \caption{Example of usage of the order independant transparency.}
    \label{order}
\end{figure}

In this \textit{fragment shader}, we setup the technique of \textit{order-independent transparency} that consists to have in each pixel of the screen an array of color and depth (in our case the depth of the vertex where teh splat is anchored) in order to know which pixel is in front of the others pixels. In our example in the figure \ref{splatting_principle}, we have some splats that cover others splats. How do we know which one is in front the other if we treat them independently? For example, if there is a pixel where splat cover another one, we construct an array with 2 elements composed of a color and a depth so we can know which one is in front of the other one (see example in the figure \ref{order}).

\chapter{Results and performance}

%%%% results
% several example with differents splats
% use noise to change the size of the splats in the object
% use tangents/normals map to rotate the splats
% differents shading



%%%% PERF
% nom de la carte
% resolution de l'image
% non-optimized

\chapter{Conclusion and discussion}

\subsection{Conclusion}



\subsection{Discussion}

%\include{./chapters/7_Appendix}
%=========================================================


%=========================================================
\backmatter

\bibliographystyle{plain} % plain-fr si rapport en français
\bibliography{proceduralBib,customBib}

%\cleardoublepage % Goes to an odd page
%\pagestyle{empty} % no page number
%~\newpage % goes to a new even page

\end{document}
