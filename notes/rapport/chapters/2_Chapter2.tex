\chapter{Previous Work}

Image stylization has been around for years. Algorithms were created to automatise this desire to stylize. Some techniques use line extraction algorithm to then use convolution of points to make hand drawing styles. Hertzmann with his \textit{curve stroke} algorithm \cite{rosin_stroke_2013}  succeed to create images that look like a traditional painting with paintbrushes. To do so he computes many control pint on the original image to further place strokes. But these create a problem when we wanted to stylize videos because it treats frames independantly and so it creates bad \textit{motion continuity}. The movie \textit{Loving Vincent}\cite{LovingVincent} can illustrate what can happen in this case of bad \textit{motion continuity}.

Then some researches have be to propose a solution to this issue\cite{litwinowicz_processing_1997, hays_image_2004, bousseau_video_2007, lin_video_nodate}. The solution of Lin et \textit{al.} \cite{lin_video_nodate} is to create a segmentation manually of each key frame and then for each part of this segmentation they compute the motion. With this motion they adapt the stroke based rendering of the next frames. To have a watercolor stylization on a video Bousseau et \textit{al.} compute a texture advection to apply to the final image the wanted effect.

In our approach, the goal is to make stylized rendering of 3D objects. There are two moments in a pipeline rendering when we can stylize an object, the first is when we manipulate the vertices and the color of each triangle it is the \textit{object space}. The second is when we do the compositing with the textures that we have like shadow map, image filter, ... (manipulation of pixels of the screen) it is the \textit{image space} and also called \textit{screen space}.

\section{Object Space}

One of the most used way to colored object in 3D is the \textit{texture mapping} \cite{texture_mapping}. There are many mapping functions: flat mapping, cylindrical mapping, spherical mapping, cube mapping and the most common used the UV mapping. In UV mapping, for each vertex of the 3D object, there is a vector of texture coordinates (also called UV coordinates) that correspond to the position of a pixel in your texture (usually a 2D image). This pixel will gives the color to display for this vertex. This technique is very used in video games because it is simple to implement, it can be implement for GPU and it needs low computation. But the problem of this approach in stylization is because the mapping is done in object space we have this effect of depth on the scene and this impression that the object is in 3 dimensions and it is hard to add small details like hair, feathers or leaves. This gives a bad \textit{flatness}. On the other hand, this way of stylizing gives a good \textit{temporal continuity} \cite{benard_dynamic_2009} because every vertex has his color in the texture so when the object in moving the colors displayed move in the same way. They also use an another technique to improve the \textit{temporal continuity} Bénard et \textit{al.} \cite{benard_dynamic_2009} use procedural fractal noises as textures that he adapts with the distance from the camera. This fractal noise can create an effect of inifinite zoom (like in this example: \href{https://www.shadertoy.com/view/XlBXWw?fbclid=IwAR1fU2JxQzXtks1ZcmVmzrHiv646G8w2gWceeiV-UToeFkAFMQ2NecbsGGs}{ShaderToy}), it can be helpful in stylization to avoid problem of pixellisation when the scale of the rendered object is increase or decreased. These two approaches are used in our work in order to get a better \textit{temporal continuity}.


As in real painting, some techniques to stylize is to draw elements often they are strokes and sometimes they are dots and with the convolution of dots it creates lines. Overcoat\cite{schmid_overcoat:_2011} choose to draw strokes on a 3D model, this is an interactive software to help artist. It has 3 tools, the hair tool that permit to draw starting in same direction of the normal at any of point of the surface, the feather tool that works the same but with the tangent of the surface and the level set tool that permit to draw at a certain distance of the object but keeping the curvature of the surface.

\section{Image Space}
