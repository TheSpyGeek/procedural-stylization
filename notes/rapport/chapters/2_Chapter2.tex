\chapter{Previous Work}

Image stylization has been around for years. Algorithms were created to automatise this desire to stylize. Some techniques use line extraction algorithm to then use convolution of points to make hand drawing styles. Hertzmann with his \textit{curve stroke} algorithm \cite{rosin_stroke_2013}  succeed to create images that look like a traditional painting with paintbrushes. To do so he computes many control pint on the original image to further place strokes. But these create a problem when we wanted to stylize videos because it treats frames independantly and so it creates bad \textit{motion continuity}. The movie \textit{Loving Vincent}\cite{LovingVincent} can illustrate what can happen in this case of bad \textit{motion continuity}.

Then some researches have be to propose a solution to this issue\cite{litwinowicz_processing_1997, hays_image_2004, bousseau_video_2007, lin_video_nodate}. The solution of Lin et \textit{al.} \cite{lin_video_nodate} is to create a segmentation manually of each key frame and then for each part of this segmentation they compute the motion. With this motion they adapt the stroke based rendering of the next frames. To have a watercolor stylization on a video Bousseau et \textit{al.} compute a texture advection to apply to the final image the wanted effect.

In our approach, the goal is to make stylized rendering of 3D objects. There are two moments in a pipeline rendering when we can stylize an object, the first is when we manipulate the vertices and the color of each triangle it is the \textit{object space}. The second is when we do the compositing with the textures that we have like shadow map, image filter, ... (manipulation of pixels of the screen) it is the \textit{image space} and also called \textit{screen space}.

\section{Object Space}

One of the most used ways to colored object in 3D is the \textit{texture mapping} \cite{texture_mapping}. It consists to add information to each vertex of the 3D object. These information many times are 2D coordinates that correspond to the position of a pixel in the 2D texture. This technique is very used in video games because it is simple to implement, it can be implemented for GPU and it needs low computation. As said by Bénard et \textit{al.} \cite{benard_dynamic_2009} textures naturally ensure \textit{motion coherence}. Indeed because each vertex has his color and so the color in moving with the object. For this advantage, many attempts were done to stylize with this method\cite{praun_real-time_2001, klein_non-photorealistic_2000, benard_dynamic_2009, benard_dynamic_2010}. \textit{Texture mapping} in object space gives good \textit{motion coherence} but gives a bad \textit{temporal continuity} because if the object gets bigger and bigger, pixelization will appear. In order to solve this problem some\cite{klein_non-photorealistic_2000, benard_dynamic_2009} tries to use mipmaps (combining multiple scales of textures) to improve \textit{temporal continuity}. Bénard et \textit{al.}\cite{benard_dynamic_2010} use the same principle but with procedural textures. They create multiple noises with different frequency and combine them playing with transparency. Moreover, they overlap the noise to make an impression of infinite zoom effect (like in this example: \href{https://www.shadertoy.com/view/XlBXWw?fbclid=IwAR1fU2JxQzXtks1ZcmVmzrHiv646G8w2gWceeiV-UToeFkAFMQ2NecbsGGs}{ShaderToy}). With this method patterns of the texture have an almost constant size regardless of the size of the object.

\newpage


As in real painting, some techniques to stylize is to draw elements often they are strokes and sometimes they are dots and with the convolution of dots it creates lines. Overcoat\cite{schmid_overcoat:_2011} choose to draw strokes on a 3D model, this is an interactive software to help artist. It has 3 tools, the hair tool that permit to draw starting in same direction of the normal at any of point of the surface, the feather tool that works the same but with the tangent of the surface and the level set tool that permit to draw at a certain distance of the object but keeping the curvature of the surface. This technique has a good \textit{temporal continuity} and on the results they present it seems to have a good \textit{motion coherence} and a good \textit{flatness} but these results were do manually by artist.

The problem of the \textit{texture mapping} in object space is that it gives a bad \textit{flatness} due to the distance of the camera of each vertex.

\section{Image Space}

We can also obtain stylized images by manipulating only its pixels. In MNPR\cite{montesdeoca_mnpr:_2018} a framework for real-time expressive non-photorealistic rendering they use procedural noise to modify the density of pigments on their images to make an effect of rendering on a real paper sheet. Bousseau et \textit{al.}\cite{bousseau_video_2007} use bidirectional texture advection in order to make a watercolor style.
