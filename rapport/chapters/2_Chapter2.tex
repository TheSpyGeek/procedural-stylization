\chapter{Previous Work}

Many research has been done in stylizing 3D scenes\cite{schmid_overcoat:_2011, praun_real-time_2001, klein_non-photorealistic_2000, benard_dynamic_2009, benard_dynamic_2010, freudenberg_walk-through_2001, benard_state---art_2011} trying to propose solutions or trade-offs for the problem of \textit{temporal coherence}. In this section of this report, we will present techniques to stylize 3D scenes and we will show their advantages and disadvantages. In order to render an image of a 3D object in the screen, a graphics program goes through several steps that compute different information like the gradient of the image, the shadows made by the object, the amount of light received by the object, etc. The gathering of all these steps is called \textit{graphical pipeline rendering}. In this graphical pipeline, there are two locations where we can stylize the objects. The process of stylizing can be applied either on the surface of the 3D objects (\textit{object-space}) or on the pixels of the screen after the models have been rendered (\textit{image-space}). We will treat these two spaces separately and with the two different types of methods to stylize.



\section{Object Space}


\begin{figure}[H]
    \begin{center}
    \includegraphics[scale=0.4]{pics/texture_mapping.png}
    \end{center}
    \caption{Texture mapping: example}
    \label{texture_mapping}
\end{figure}

\begin{figure}[H]
    \begin{center}
    \includegraphics[scale=0.2]{images/marble.jpg}
    \end{center}
    \caption{Marble with procedural texture}
    \label{marble_rendering}
\end{figure}


\begin{figure}[H]
    \begin{center}
    \includegraphics[scale=0.2]{pics/hatching.png}
    \end{center}
    \caption{Real time hatching rendering \cite{praun_real-time_2001}}
    \label{hatching_rendering}
\end{figure}

\begin{figure}[H]
    \begin{center}
    \includegraphics[scale=0.3]{pics/watercolor_objspace.png}
    \end{center}
    \caption{Watercolor stylization with procedural textures \cite{benard_dynamic_2009}}
    \label{watercolor_rendering}
\end{figure}





In the object space, we work on the surface of the object and so we have all the knowledge about the geometry. \newline

% texture based

\textbf{Texture-based methods}


One of the most used ways to colored object in 3D is the \textit{texture mapping}. It consists in mapping an image on the object (Figure \ref{texture_mapping}\ref{hatching_rendering}). This technique is widely used in video games because it is easy to implement, it can be implemented for GPU and it needs low computation. As shown in the example, texture mapping can be used with images as a texture in order to stylize the object\cite{praun_real-time_2001, klein_non-photorealistic_2000, freudenberg_walk-through_2001}. Texture mapping can also be done with \textit{procedural noises}\cite{perlin_improving_2002} in this case called \textit{procedural textures}. Procedural textures are mathematically computed from 3D surfaces coordinate the most famous is Perlin but there exist others like Gabor noise, Worley noise, etc. With these textures, we can create images that look like marble (Figure \ref{marble_rendering}) or realist wood. Bénard et al.\cite{benard_dynamic_2009, benard_dynamic_2010} use this method of texture mapping with noise in order to make watercolor stylization (Figure \ref{watercolor_rendering}). \newline

This method has a problem of \textit{foreshortening} on the silhouettes which makes an area with more elements than in the center of the object and when zooming in/out the size of the elements in the mapped texture vary depending on the distance from the camera, if we get closer to the object the elements on the texture become bigger and bigger going up to pixelization sometimes. These two problems make texture mapping bad in term of \textit{flatness} because an artist does not draw bigger strokes if the object is close and does not draw more details at the silhouettes than in the center of his objects. He is constraint by the size and shape of the tools. Texture mapping provides a wide variety of styles but limited to the shape of the object, you cannot, for instance, do a rendered image that looks like painted with brushes and especially you cannot change the shape of the object in the rendered image, a perfect sphere will always appear as a perfect sphere. On the other hand, this method naturally ensures \textit{motion coherence} and \textit{temporal continuity} because the texture is mapped directly on the surface of the object. \newline



% mark based

\begin{figure}[H]
    \begin{center}

    \includegraphics[scale=0.5]{images/overcoat.jpg}
    \end{center}
    \caption{OverCoat: implicit canvas \cite{schmid_overcoat:_2011}}
    \label{overcoat_figure}
\end{figure}

\textbf{Mark based methods}

The natural way to stylize 3D objects is to as an artist apply paint strokes on the object. These paint strokes can be represented with smalls images also called splats. Some work \cite{meier_painterly_1996, Fekete_2000, chi_stylized_2006}(more in the state of the art \cite{benard_state---art_2011}) use point distributions in order to make anchor points for splats. These point distributions are often computed in image space and then are projected on the model. Anchor these splats to the model improve the \textit{motion coherence} because each splat will follow the motion of the 3D model. These splats are rendered in the image space as 2D sprites, it preserves the \textit{flatness}. The problem is how to have the point distribution and how can we control it in order to have a uniform, not too sparse and not too dense distribution. Moreover, these point distribution does not provide control over the \textit{temporal continuity}. In our method, we use procedural noise to anchor the splats.

Daniels\cite{Daniels_1999} and Schmid\cite{schmid_overcoat:_2011} propose to project splat on an inflated proxy geometry and stored them in nodes but anchored to the geometry of the model but this technique is expensive in term of storage. OverCoat\cite{schmid_overcoat:_2011} is an interactive software that permit to the user to stylize manually a 3D object (\href{https://www.youtube.com/watch?v=Nfw6JxC9Fw4}{Overcoat demo}). Their techniques are the closer techniques to our proposed solution. In their solution, they use 3D splines drawn by the user and they display images along these lines. Each spline is stored in nodes that contains the image to splat along this line, the color, the thickness and the transparency that permit the user to edit the stylization after drawing the splines. These splines are anchored on a proxy generated by the program this has the disadvantage of placing the stylization at the beginning of the graphical pipeline. It has a problem of \textit{foreshortening} of the silhouettes that make the rendering blurred at some place of the image. Overcoat has a big problem of flatness when rescaling the stylized object, this is because the user draws marks at a specific scale so when we zoom in the scene the marks looks bigger or smaller and that affect the \textit{flatness} of the rendered image. All the marks drawn are anchored to the geometry so it ensures the \textit{motion coherence} and because they use a proxy geometry the shape of the object can be changed, some details can be added. If the object is animated it will create a hole or gathering of marks.

The mark based methods have a problem of foreshortening has said before and the non-constant size of the marks when rescaling of the object make the impression of bad \textit{flatness}. But the \textit{motion coherence} is ensured by anchoring the marks to geometry that makes them follow the motion of the stylized object.



% IMAGE SPACE

\section{Image space}

In image space, we work on images with 2d coordinates that means working on the pixel of the image directly. In a normal pipeline rendering, the image space correspond to the \textit{compositing} which is the moment when many information have already be rendered and are stored in images (like shadow map, image filter, global illumination, etc.), during the \textit{compositing} we combine the computed images to render the final image. \newline

\textbf{Texture-based methods}

\begin{figure}[H]
    \begin{center}

    \includegraphics[scale=0.5]{pics/watercolor_MNPR.png}
    \end{center}
    \caption{Image filter for watercolorization \cite{montesdeoca_mnpr:_2018}}
    \label{watercolor_MNPR}
\end{figure}

Many methods to stylize in image space used texture based approaches. It consists applying the texture to the entire image (Figure \ref{watercolor_MNPR}) \cite{benard_state---art_2011, montesdeoca_mnpr:_2018} but in the case of stylizing animated scenes, the problem is how do we deform the texture to minimize the apparition of sliding artefacts. Sliding artefacts are some parts of the image that do not move according to scene motion and give the impression that they are sliding. We can distinguish two families of approaches to solve this problem. The first family of approaches use an approximation of the 3D camera motion with 2D transformations of the texture\cite{cunzi_dynamic_nodate}. This gives a nice trade-off between \textit{motion coherence} and \textit{flatness} but it is limited to static scenes and a set of few camera motions. Moreover, sliding artefacts still occur with strong parallax so Fung et al.\cite{fung_pen-and-ink_nodate} and Breslav et al.\cite{breslav_dynamic_nodate} improve the approximation of the scene motion in order to reduce sliding artefacts.

The second family of approaches use deformations to animate the texture\cite{bousseau_video_2007}. These deformations are computed from the optical flow of a video. This is an extension of the methods used in vector field visualization by Neyret\cite{neyret_imagis-gravir_nodate}. These deformations can distort the texture and alter the original pattern. The method of Bousseau et al.\cite{bousseau_video_2007} is very effective with stochastic textures as the fractalization process but creates artefacts with structured patterns. \newline

These texture based methods for 3d scenes have issues with \textit{motion coherence} and \textit{temporal continuity} because of the independence of each frame and also these techniques suffer to make a poor variety of style, you cannot do for example hairs, brushes painting, pencil drawings. But they can do some different styles with good \textit{flatness}.\newline

\textbf{Mark based methods}

\begin{figure}[H]
    \begin{center}

    \includegraphics[scale=0.5]{pics/hertzmann_algo.png}
    \end{center}
    \caption{Hertzman's algorithm on an image \cite{rosin_stroke_2013}}
    \label{Hertzman_algo}
\end{figure}

\begin{figure}[H]
    \begin{center}

    \includegraphics[scale=0.5]{pics/segmentation_example.png}
    \end{center}
    \caption{Example segmentation image\cite{benard_state---art_2011}}
    \label{segmenation_example}
\end{figure}

A method used to stylize in image space consists drawing strokes/splats at some place of the image\cite{bleron_motion-coherent_2018, vergne_implicit_2011, benard_active_nodate, zeng_image_2009, grabli_programmable_2010}. It is like adding other images in an image (like you can do in Photoshop or Gimp, Figure \ref{Hertzman_algo}). The question of these mark based methods is where do we place these marks in order to have a stylized rendering without losing the meaning of the scene and without losing details. A first approach is to extract lines that are relevant like the silhouettes, etc. \cite{vergne_implicit_2011, grabli_programmable_2010, lee_line_nodate} and then stylize the image with this information, like keeping only the extracted lines and change the shape of each line or apply strokes along these lines as Vergne et al.\cite{vergne_implicit_2011} did try to have a good \textit{temporal coherence}. The problem of these techniques is the popping marks due to a bad \textit{temporal continuity} because we do not know if in the next frame there will be more details or not or the color to display is different from the current one.
A second approach is to segment the image (Figure \ref{segmenation_example}) in order to have the different parts of the scene\cite{zeng_image_2009, lin_video_nodate}. Thanks to this segmentation, they apply different strokes for each part of the image with the corresponding colors. The work of Lin et al.\cite{lin_video_nodate} is about videos so they use the optical flow of the videos in order to have a good \textit{temporal coherence}. These mark based methods have a good impression of \textit{flatness} thanks to the splatting in image space, this is something that we will use in our approach.

These methods of splatting have almost the same problem that the texture-based method they suffer of bad \textit{motion coherence} and \textit{temporal continuity}. Temporal continuity that can be improved using a motion flow which cannot be possible in our case. In the other hand, changing the type of splat, the size and their orientation we can make many different styles, like brushes painting, pencil drawings, stippling, hairs. These methods draw 2d strokes/brushes in the image so it looks like drawn by a person on a flat surface which makes a good \textit{flatness} impression. \newline


% Conclusion state of the art

% mark based method propose de variation de style
% l'espace objet permet une bonne coherence de mouvement
%

The figure \ref{tableau_comparatif} is a summary of the previous works. The interesting point is mark-based method can make a wider variety of styles than the texture-based method. We can see that for \textit{motion coherence} and \textit{temporal continuity} doing the stylization in object space is better than in image space but in image space, we can do stylization that has a good \textit{flatness} impression contrary to the object space.


\begin{figure}[H]

    \begin{tabular}{|l|*{4}{c|}}
    \hline
         & \textbf{Motion coherence} & \textbf{Flatness} & \textbf{Temporal continuity} & \textbf{Style variation} \\
    \hline
    \textbf{Object space} & & & & \\
    \hline
    Texture-based methods & ++ & - - & ++ & - - \\
    \hline
    Mark based methods & ++ & +/- & +/- & +/- \\
    \hline
    \textbf{Image space} & & & & \\
    \hline
    Texture-based methods & -  & ++ & - & - - \\
    \hline
    Mark based methods & - & ++ & - - & + \\
    \hline
    \end{tabular}

    \caption{Summary of trade-offs made in different approaches}
    \label{tableau_comparatif}
\end{figure}
