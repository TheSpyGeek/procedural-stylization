\chapter{Discussion and conclusion}

\subsection{Discussion}

\textit{Watercolorization}. We tried to render images imitating the effect of watercolor but we think that it needs to create procedural marks and to have more control on the transparency of these splats which is possible with some modifications in our method. \newline

\textit{Depth in the splat itselft}. The splats are 2d images so when we draw them on the screen they keep their 2d aspect. But with the arrays needed for the \textit{order-independent transparency} we can control the depth that we put in these arrays. So we imagine that we can make volumic splat (3d splats) in order to make for example hairs that intertwine. \newline

\textit{Level of details}. In our method, we use a global procedural noise to control the anchor points of the marks but we could combine more procedural noises in order to add more details in some parts of the object or as an artist will do add more details in the silhouettes. \newline

\subsection{Conclusion}

We presented a post-processing stylization pipeline that draws marks over the screen while keeping a good \textit{motion coherence}. Our method allows the user to choose the density of marks, the content of marks, their orientation and their size. Our main contribution in this report is the work on the Worley noise that permits to have a controllable texture which is getting closer to a point distribution with control of the size of points. The fractalization of the noise is not our contribution but we adapted it to our wanted, adding the choice of how many noises we mix and with which function interpolate the value. This solution only requires G-buffer data allowing it to be embedded in the compositing stage of most rendering software. This technique was then put to the test in order to make several images. We intend to reproduce existing styles which works with some, but some need modifications in our pipeline that can be done in future works.
