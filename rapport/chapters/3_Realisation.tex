\chapter{Realisation}


\section{Overview}

\begin{figure}[H]
    \begin{center}
    \includegraphics[scale=0.4]{images/Overview2.png}
    \end{center}
    \caption{Overview of our solution.}
    \label{overview}
\end{figure}

As explained in the state of the art and in the figure\ref{tableau_comparatif} each approach has its advantages and its disadvantages. Our main idea is to use procedural texture rendered in the object space to guide the splats in image space.  This solution permits to apply something like 2D images on the screen so have a good \textit{flatness} while keeping the information on the silhouettes, the orientation, the depth, etc. This solution only needs images as input so it can easily be integrated into most of rendering pipelines without re-rendered images. In the figure \ref{overview}, there is an overview of our pipeline. First we computed all the data  in what we called G-Buffer (normals, tangents, position, matrices, depth, UV coordinates) then we render a color (in this example a simple shading) after this we computed our procedural noise that will help to anchor the splats to the object and we choose the splat that will convolves in order to make the stylization. Finally, we draw splat on the entire screen, then we decide which one has to be displayed this is the splatting step and the last step is the blending which computed which splat is in front of the others and which color has to be prompted. \newline

We chose to use mark based methods to stylize our scene because texture-based methods in image space give a poor variety of styles as said in the work of Bénard et al.\cite{benard_dynamic_2009}. This mark-based method implies to decide where in the image the splats will be drawn. In our problem, the goal is to anchor these splats with the objects in order to have the same motion for the splats and the object. This avoid the problem of \textit{shower door effect} and ensure good \textit{motion coherence}. So we needed anchor points depending on the position of our object. Therefore in our approach, we used a procedural noise\cite{perlin_improving_2002} as a texture our 3D object. The procedural noises are easy to implement, fast to compute and easy to manipulate. Like every texture computed in object space, it has a good motion coherence.


\section{Procedural noise and fractalization}

\begin{figure}[H]
    \begin{center}
    \includegraphics[width=40mm, height=40mm]{images/PerlinNoise2d.png}
    \includegraphics[width=40mm, height=40mm]{images/GaborNoise2d.png}
    \includegraphics[width=40mm, height=40mm]{images/WorleyNoise2d.jpg}
    \end{center}
    \caption{Examples of procedural texture: \textit{left using perlin noise, middle using Gabor noise,right using worley noise}.}
    \label{procedural_texture}
\end{figure}

% Description of procedural noise

We compute procedural texture in order to create anchor points. In computer graphics, they are usually used to create more complex rendering like the marble texture (figure \ref{marble_rendering}). These textures are computed from procedural noise with a mathematical process with only the position as a parameter. The algorithm to create these textures are implicit that means that with only one position you can have the value without knowing the rest of the texture. That is why they are very used is computer graphics because they can be computed in parallel on GPU. There exist much procedural noises such as Perlin noise, Worley noise, Gabor noise, Value noise, Gradient noise, etc. In order to map the procedural texture to the 3d object, we compute it with the vector position of each vertex of the object.

% how we use it

\begin{figure}[H]
    \begin{center}
    \includegraphics[scale=0.6]{images/noise/addition.png}
    \end{center}
    \caption{Usage of procedural texture to anchor splats: (left: splat image, middle: procedural teture, right: rendered image, bottom: color).}
    \label{procedural_noise_anchor}
\end{figure}

In our case, we used the procedural texture to anchor the splat. For each pixel of the final image we "paste" a splat. Each splat as a weight that will correspond to its opacity during the splatting and the blending. The value of this weight corresponds to the value of the noise on the current point. In the figure \ref{procedural_noise_anchor}, you can see in the areas where the noise is close to zero there is no splat displayed in the final image. Thanks to this mechanism we can control the density of splat in the image. In our case, we worked mainly with the Worley that when inverted almost correspond to point distribution.



% Worley

\begin{figure}[H]
    \begin{center}
    \includegraphics[scale=0.2]{images/noise/worley_explain.png}
    \end{center}
    \caption{Worley noise in 2d.}
    \label{worley_explain}
\end{figure}

We mainly work with the Worley noise which is a cellular noise as you can see at the right in the figure \ref{procedural_texture}. To create a procedural texture from a Worley noise, we divide the image in cells (squares with the same size in 2d, red lines in the figure \ref{worley_explain}), then in each square we compute a point with a seeded random (white dot in the figure \ref{worley_explain}) and finally at each pixel of the texture we put as value the distance with the closest point computed previously with the random seed. Increase the frequency of this noise correspond to divide the image with smaller cells and so have more cells in the image. We explained the principle for a 2d texture but it is the same principle in 3d with a volumic grid. In our pipeline, we use the inverse of this noise (1-worleynoise) in order to have the white pixels of the procedural texture as anchor points. We add a threshold parameter to compute the noise in order to reduce the number of splat to display but especially to have small points in the procedural texture if not the splats convolve as you can see in the rendered image in the figure \ref{procedural_noise_anchor}.

\begin{figure}[H]
    \begin{center}
    \includegraphics[scale=0.3]{images/noise/worley3d_thresh_1.png}
    \includegraphics[scale=0.3]{images/noise/worley3d_thresh_2.png}
    \includegraphics[scale=0.3]{images/noise/worley3d_thresh_3.png}
    \end{center}
    \caption{Use of threshold in the case of Worley noise with the same frequency.}
    \label{worley_threshold}
\end{figure}

As we said before we use a threshold for the purpose of having another control on the noise. With the Worley noise, this threshold corresponds to the "size" of the points. If we increase the threshold the points become smaller as you can see in the figure \ref{worley_threshold}. In order to facilitate the control of this noise, we adapt this threshold according to the frequency and the distance from the camera. That means that the size of each point is constant if the frequency or/and the distance from the camera is changed. \newline

% Add fractalization


\textbf{Fractalization}

\begin{figure}[H]
    \begin{center}
    \includegraphics[scale=0.3]{images/fractalization_principle.png}
    \end{center}
    \caption{Principle of fractalization \cite{benard_dynamic_2010}.}
    \label{fractalization_principle}
\end{figure}

\begin{figure}[H]
    \begin{center}
    \includegraphics[scale=0.6]{images/fractal_explained.png}
    \end{center}
    \caption{Combination of the noises with the weight value in this case with 2 noises.}
    \label{fractalization_practical}
\end{figure}


Bénard et \textit{al.}\cite{benard_dynamic_2010} introduce the mechanism of fractalization on textures. This technique creates an impression of infinite zoom effect (like in this example: \href{https://www.shadertoy.com/view/XlBXWw?fbclid=IwAR1fU2JxQzXtks1ZcmVmzrHiv646G8w2gWceeiV-UToeFkAFMQ2NecbsGGs}{ShaderToy of Neyret Fabrice}). This is helpful in our case because when the camera is getting closer and closer or further and further less or more points in the texture. With this mechanism, we can get a quasi-constant number of points on the screen. In our method of stylizing it improves the \textit{temporal continuity} because there is always display even if you get very close to the object and it also improves the \textit{flatness} impression because there is almost the same number of anchor points in the screen and their size is quasi-constant. This method alters the original pattern of the texture as you can see in the figure \ref{fractalization_principle} it can be an issue because some pattern cannot be fractalized (like the checkboard pattern) but it works well with stochastic textures. To creates this fractalization, they use textures at multiple frequencies (not necessary procedural texture) (see figure \ref{fractalization_principle}) and they combine them with the transparency and they overlap them according to the distance from the camera as you can see in the results of the figure \ref{fractalization_principle}. \newline


The figure \ref{fractalization_principle} show 4 noises (octaves) at different frequency it is a zooming cycle. As you can see, if the distance from the camera is close to 0 the frequency of the noises is high and if we move away from the object the frequency decrease. So when the distance from the camera is twice as large the frequency of the noise displayed is multiplied by 2. During the zoom, we modulate each octave with a weight. To compute the weight of each noise we use Gaussian interpolation centered on our current distance from the camera (see figure \ref{fractalization_practical}) with a sigma depending on the number of noises that we want to combine.







 % Bénard et \textit{al.}\cite{benard_dynamic_2010} use the same principle but with procedural textures. They create multiple noises with different frequency and combine them playing with transparency. Moreover, they overlap the noise to make an impression of infinite zoom effect (like in this example: \href{https://www.shadertoy.com/view/XlBXWw?fbclid=IwAR1fU2JxQzXtks1ZcmVmzrHiv646G8w2gWceeiV-UToeFkAFMQ2NecbsGGs}{ShaderToy}). With this method patterns of the texture have an almost constant size regardless of the size of the object but it can create small problems of \textit{temporal continuity}. In our method, we will use this technique of fractalization of a procedural noise. \newline

\section{Splatting}

\begin{figure}[H]
    \begin{center}
    \fbox{\includegraphics[width=30mm, height=30mm]{images/splats/dot_splat.png}}
    \fbox{\includegraphics[width=30mm, height=30mm]{images/splats/hair_splat.png}}
    \fbox{\includegraphics[width=30mm, height=30mm]{images/splats/line_splat.png}}
    \fbox{\includegraphics[width=30mm, height=30mm]{images/splats/paint_splat.png}}
    \end{center}
    \caption{Example of what the splats can be.}
    \label{splat_examples}
\end{figure}

The marks based method try to imitate how the artists do when they stylize. They draw on a flat surface that gives a good impression of \textit{flatness}. We use the same principle to stylize our scene. We put splats/marks directly on the image like an artist will do (illustrated in the figure \ref{procedural_noise_anchor}). \newline

In our solution, the user controls the splat image he can use anything. These marks can be leaves, hairs, dots, feathers, lines, paint brushes, etc. They can also be created from a procedural noise. The user has also the control on the size of these splats, he has global control of the size and control on a specific part of the object through a texture. The rotation of the splats is also a parameter that the user can control. He can choose to rotate all the splat, for example, depending on the normals or depending on the tangents. \newline

Artists control how their marks are combined for example if he is doing painting he could want to have non-transparent marks in order to cover the marks behind the new one. He could want more transparency if for example, he wants to do watercolorization.\newline

In our method, we take this into account during the blending of all marks. The user can control how many marks the stylization will mix and control the transparency (alpha blending). For example, the user can choose to only show the marks at the top of the flat surface (the closest to the camera) or he can choose to mix some marks in order to account some marks covered.


\section{Stylization}

As explained in the previous section, in our method, the user controls the content of the splats, their size, their orientation, the density of splat on the object and how they are blended with the alpha blending. In this section, we will show you some examples of how we did specific style using our pipeline. \newline

\textbf{Pointillism}

\begin{figure}[H]
    \begin{center}
    \includegraphics[width=80mm, height=80mm]{Resultats/spotPoint/final.png}
    \end{center}
    \caption{Example of pointillism with our method.}
    \label{final_point}
\end{figure}

In order to make pointillism style we use a dot as splat (gaussian function in 2d, the first in the figure \ref{splat_examples}), the procedural noise to control the number of splats has a high frequency with a high threshold in order to have small anchor points, the size of the splats can vary depending on which size of brush he wants to imitate but usually we use small splat size for this style and for the alpha blending it usually adapt to not mix many marks. In this case, the orientation does not matter. The figure \ref{final_point} shows the final rendered image. \newline

\textbf{Brush painting}

\begin{figure}[H]
    \begin{center}
    \includegraphics[width=80mm, height=80mm]{Resultats/painting1/final.png}
    \end{center}
    \caption{Example of painting with brushes with our method.}
    \label{final_brushes}
\end{figure}

In order to make an image made like a painter we use an image of a paint strokes (like the fourth in the figure \ref{splat_examples}), we do not need too much splat displayed so the frequency of the procedural noise does not have to be very high but a threshold high/medium, the size of the splats depend of the 3d model but usually we use large splat. For the alpha blending it depends on the effect wanted, in our case, we mixed many splats. And for the orientation usually the artists align the strokes with the tangents but it works well if the strokes are horizontal. The figure \ref{final_brushes} shows the final rendered image. \newline

\textbf{Hair rendering}

\begin{figure}[H]
    \begin{center}
    \includegraphics[width=80mm, height=80mm]{Resultats/bouledepoil1/final.png}
    \end{center}
    \caption{Example of hairs rendering with our method.}
    \label{final_hair}
\end{figure}

In order to show that we have the control of the \textit{flatness} impression we tried to create hairs with a 3d aspect like it can be done in digital painting (see figure \ref{final_hair}). To do this we use a splat created mathematically that looks like a hair (the second in the figure \ref{splat_examples}) rotated according to the normals of the object. With the procedural noise that is used to anchor marks, the density can be varied impacting directly the density of hairs in the fur. The size of the splats varies depending on what size you want your hair to be. And for the alpha blending, you do not want to mix too many splats but you do not want to only show the last hairs drawn. The figure \ref{final_hair} shows the final rendered image. \newline

Here there are 3 examples of how we use the method to do some styles but the same styles can be done differently and others styles can be done as well. These 3 examples show what is possible to do with this method and show the control of the \textit{flatness} we have.
